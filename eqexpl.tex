\documentclass{article}
\usepackage[T2A,T1]{fontenc}
\usepackage[utf8]{inputenc}
\usepackage{hyperref}
\usepackage{eqexpl}

\begin{document}

\title{eqexpl}
\author{Konstantin Morenko}

\maketitle

The package is licenced under Creative Commons Attribution-ShareAlike
4.0 International (CC BY-SA 4.0)

\section{The aim of the package}

The package was developed as an answer to the question \href{https://tex.stackexchange.com/q/95838/119485}{on tex.stackexchange.com}

Package was developed to give the tool to make <<perfect>> explanation
for formulas, not just the enumeration.

This package allows to describe formula's variables in unified through
the document manner.

\section{Similar packages}

Nomencl: \href{http://ctan.org/pkg/nomencl}{http://ctan.org/pkg/nomencl}

\section{Contributors}

Konstantin Morenko \href{mailto:me@konstantin-morenko.ru}{me@konstantin-morenko.ru}

The package currently hosted on GitHub:
\href{https://github.com/konstantin-morenko/latex-equation-explanation}{https://github.com/konstantin-morenko/latex-equation-explanation}

\section{Architecture}

The list consist of few lengths:
\begin{itemize}
\item width of <<intro>> section (default is empty, 0pt);
\item width of spaces between elements (default is 2mm);
\item width of item block (default is 5mm);
\item width of separator (default is '---');
\item the rest of the width of a text (used to align left side of the
  explanation text).
\end{itemize}

\section{Usage}

First, include package into preamble with

\begin{verbatim}
  \usepackage{eqexpl}
\end{verbatim}

Then write a formula and describe the parameters

\begin{equation}
  E = m c^2
\end{equation}
\begin{eqexpl}
\item{$E$} equivalent energy
\item{$m$} mass
\item{$c$} speed of light ($ c \approx 3 \times 10^8 m/s$)
\end{eqexpl}
using
\begin{verbatim}
  \begin{equation}
    E = m c^2
  \end{equation}
  \begin{eqexpl}
  \item{$E$} equivalent energy
  \item{$m$} mass
  \item{$c$} speed of light ($ c \approx 3 \times 10^8 m/s$)
  \end{eqexpl}
\end{verbatim}

\section{Configure and examples}

\subsection{Test list}

\newcommand{\testList}{
\item{U} voltage at the section, V;
\item{Rs} total section resistance, Ohm.
\item{$Very^{46}$}very very very very very very very very very very very
  very very very very very very very very very very very very very
  very very very very very very very very very very very very very
  very very very very very very very very very long line;
}

This list is used for next examples:
\begin{eqexpl}
  \testList
\end{eqexpl}

\subsection{Setting eqexplSetSpace}

\noindent Set \verb+\eqexplSetSpace{0mm}+

\eqexplSetSpace{0mm}
\begin{eqexpl}
  \testList
\end{eqexpl}

\vspace{5mm}

\noindent Set \verb+\eqexplSetSpace{}+ (default 2mm)

\eqexplSetSpace{}
\begin{eqexpl}
  \testList
\end{eqexpl}

\vspace{5mm}

\noindent Set \verb+\eqexplSetSpace{10mm}+

\eqexplSetSpace{10mm}
\begin{eqexpl}
  \testList
\end{eqexpl}

\eqexplSetSpace{}

\subsection{Setting eqexplSetIntro}

\noindent Set \verb+\eqexplSetIntro{where}+

\eqexplSetIntro{where}
\begin{eqexpl}
  \testList
\end{eqexpl}

\vspace{5mm}

\noindent Set \verb+\eqexplSetIntro{in this equation}+

\eqexplSetIntro{in this equation}
\begin{eqexpl}
  \testList
\end{eqexpl}

\eqexplSetIntro{}

\subsection{Setting eqexplSetDelim}

\noindent Set \verb+\eqexplSetDelim{---}+ (default)

\eqexplSetDelim{---}
\begin{eqexpl}
  \testList
\end{eqexpl}

\vspace{5mm}

\noindent Set \verb+\eqexplSetDelim{=}+

\eqexplSetDelim{=}
\begin{eqexpl}
  \testList
\end{eqexpl}

\vspace{5mm}

\noindent Set \verb+\eqexplSetDelim{$\to$}+

\eqexplSetDelim{$\to$}
\begin{eqexpl}
  \testList
\end{eqexpl}

\eqexplSetDelim{---}

\subsection{Setting eqexplSetItemWidth}

\noindent Set \verb+\eqexplSetItemWidth{5mm}+ (default)

\eqexplSetItemWidth{5mm}
\begin{eqexpl}
  \testList
\end{eqexpl}

\vspace{5mm}

\noindent Set \verb+\eqexplSetItemWidth{10mm}+

\eqexplSetItemWidth{10mm}
\begin{eqexpl}
  \testList
\end{eqexpl}

\eqexplSetItemWidth{5mm}

\subsection{Setting item width for 'begin-end' block}

When we have a long variable name (for example \verb+very-very-long+),
it could lead us to overhelming the variable name as in example below

\begin{eqexpl}
  \item{$long$} just variable
  \item{$very-long$} just variable
  \item{$very-very-long$} just variable
\end{eqexpl}

User could set a parameter to environment to use custom item width for
current block in opposition to setting it before block to new value
and unsetting it to default after the end of the block.  For this
purpose use \verb+\begin{eqexpl}[width]+.

\noindent Set \verb+\begin{eqexpl}[10mm]+

\begin{eqexpl}[10mm]
  \item{$long$} just variable
  \item{$very-long$} just variable
  \item{$very-very-long$} just variable
\end{eqexpl}

\vspace{5mm}

\noindent Test for backing to default in next block

\begin{eqexpl}
  \item{$long$} just variable
  \item{$very-long$} just variable
  \item{$very-very-long$} just variable
\end{eqexpl}

\vspace{5mm}

\noindent Set \verb+\begin{eqexpl}[20mm]+

\begin{eqexpl}[20mm]
  \item{$long$} just variable
  \item{$very-long$} just variable
  \item{$very-very-long$} just variable
\end{eqexpl}

\vspace{5mm}

\noindent Test for backing to default in next block

\begin{eqexpl}
  \item{$long$} just variable
  \item{$very-long$} just variable
  \item{$very-very-long$} just variable
\end{eqexpl}

\subsection{Setting eqexplItemAlign}

\noindent Set \verb+\eqexplSetItemAlign{r}+ (default)

\eqexplSetItemAlign{r}
\begin{eqexpl}
  \testList
\end{eqexpl}

\vspace{5mm}

\noindent Set \verb+\eqexplSetItemAlign{l}+

\eqexplSetItemAlign{l}
\begin{eqexpl}
  \testList
\end{eqexpl}

\vspace{5mm}

\noindent Set \verb+\eqexplSetItemAlign{c}+

\eqexplSetItemAlign{c}
\begin{eqexpl}
  \testList
\end{eqexpl}

\eqexplSetItemAlign{r}

\end{document}
